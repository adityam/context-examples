\startbuffer[references]
  @electronic{LuaTeX-1.0.3,
    title = {LuaTeX 1.0.3 announcement},
    author = {Hagen, Hans and Scarso, Luigi and Hoekwater, Taco},
    year = {2017},
    howpublished = {\url{https://tug.org/pipermail/luatex/2017-February/006345.html}},
  }
  
  @electronic{LuaJITFFI,
    title = {LuaJIT: FFI Library},
    author = {Pall, Mike},
    year = {2005--2017},
    howpublished = {\url{http://luajit.org/ext_ffi.html}},
  }

  @electronic{LuaFFI,
    title = {LuaFFI},
    author = {McKaskill, James R.},
    year = {2011},
    howpublished = {\url{https://github.com/jmckaskill/luaffi}},
  }
  
  @book{GSL,
    title = {GNU Scientific Library Reference Manual},
    author = {Galassi, M. and others},
    edition = {3},
    year = {2009},
    url = {http://www.gnu.org/software/gsl/},
  }

  @electronic{qag,
    title = {GNU Scientific Library: Numerical Integration},
    author = {{The GSL Team}},
    year = {1996--2017},
    howpublished = {\url{https://www.gnu.org/software/gsl/doc/html/integration.html}},
  }

  @book{QUADPACK,
    title = {Quadpack: A Subroutine Package for Automatic Integration},
    author = {Piessens, R., Doncker-Kapenga, E. de, Überhuber, C.W., Kahaner, D.K.},
    publisher = {Springer},
    year = {1983},
  }

  @electronic{Rsvg,
    title = {Answer to \quote{How to include SVG diagrams in LaTeX?}},
    author = {Menke, Henri},
    year = {2017},
    howpublished = {\url{https://tex.stackexchange.com/a/408014}},
  }

  @electronic{FermiDiracIntegral,
    title = {Answer to \quote{Plot complete Fermi-Dirac integral in Lualatex}},
    author = {Menke, Henri},
    year = {2017},
    howpublished = {\url{https://tex.stackexchange.com/a/403794}},
  }

  @electronic{Julia,
    title = {Interfacing \LUATEX\ with Julia},
    author = {Mahajan, Aditya},
    year = {2017},
    howpublished = {\url{https://adityam.github.io/context-blog/post/interfacing-with-julia/}},
  }

  @electronic{pgf-fpu,
    title = {Floating Point Unit Library},
    author = {Feuersänger, Christian},
    year = {2015},
    howpublished = {\url{https://ctan.org/pkg/pgf}},
  }
\stopbuffer

\usemodule[tugboat]

\usemodule[pgfplots]
\pgfplotsset{compat=newest}

% We need these global declaration, so we can scatter the Lua bits
% over several buffers.
\startluacode
ffi = require("ffi")
gsl = ffi.load("gsl")
\stopluacode

\definefontfeature
  [default][default]
  [protrusion=quality,
   expansion=quality]

\setupbodyfont[modern]

\setupalign[hz,hanging]

\setuptyping
  [before={\blank[halfline]},
   after={\blank[halfline]}]

% Comment this line for a b/w version
\setupcolors[conversion=no]

\setupmathematics[integral=nolimits]

\usebtxdataset[references.buffer]
\usebtxdefinitions[aps]
\setupbtxlist[align={hz,hanging,tolerant}]

\setvariables
  [tugboat]
  [type=article,
   columns=yes,
   title={Tutorial: Using external C libraries with the \LUATEX\ FFI},
   author={Henri Menke},
   address={9016 Dunedin\\New Zealand},
   email={henrimenke@gmail.com}]

\StartAbstract
  The recent 1.0.3 release of \LUATEX\ introduced an \acro{FFI}
  library (Foreign Function Interface) with the same interface as the
  one included by default in the \LUAJIT\ interpreter.  This allows
  for interfacing with any external library which adheres to the C
  calling convention for functions, which is pretty much everything.
  In this tutorial I will present how to interface with the \acro{GNU}
  Scientific Library (\acro{GSL}) to calculate integrals numerically.
  As a showcase I will plot a complete Fermi-Dirac integral using the
  \type{pgfplots} package.  To understand this article the reader
  should have good knowledge of the Lua and C programming languages
  and a basic understanding of the C memory model.
\StopAbstract

\starttext
\StartArticle

  \footnotetext{Thanks to Hans Hagen for very useful discussions.}

  \startsection[title={The FFI library}]

    Lua is known for its rich C \acro{API} which allows interfacing
    with system libraries in a straightforward fashion.  The workflow
    for that is always the same: Write a function in C which fetches
    the arguments from the stack of the Lua interpreter and converts
    them into fixed C types.  Using the fixed-type variables call the
    library function and receive a result, either as a return value or
    as an output argument.  The result has to be converted back to a
    Lua type and pushed onto the stack of the Lua interpreter.  Then
    hand the control back to the interpreter.

    As we can already see, this recipe involves writing a lot of code,
    most of which is pure boilerplate.  Wouldn't it be great if there
    was something which would just do all the type conversion work for
    us?  And indeed there is, the \acro{FFI} \citation[LuaTeX-1.0.3,
      LuaJITFFI, LuaFFI].  The concept of a Foreign Function Interface
    is not exclusive to Lua and also exists in other languages, e.g.\
    with the \type{ctypes} library for Python.

    Different \acro{FFI}s have different ways of binding library
    functions.  The Lua \acro{FFI} chooses to parse plain C
    declarations.  The advantage of this is that when interfacing with
    C libraries, you can copy and paste function prototypes from the
    corresponding header files.  Of course, the disadvantage is, that
    for non-C libraries one has to come up with those prototypes
    oneself, which is not always an easy task.  The \acro{FORTRAN}
    language for example does not use the C-style \emph{call by value}
    convention but always uses \emph{call be reference}; that is to
    say, all types from a C function prototype would have to be
    converted to pointer types.

  \stopsection

  \startsection[title={The GNU Scientific Library}]

    The \acro{GNU} Scientific Library (\acro{GSL}) \citation[GSL] is a
    software library for scientific computing, implementing a broad
    range of algorithms.  A complete list of algorithms is way too
    long to be presented here and certainly beyond the scope of this
    tutorial.  We will only deal with the numerical integration
    routines here.

    The numerical integration routines in the \acro{GSL} are based on
    algorithms from the \acro{QUADPACK} \citation[QUADPACK] package
    for adaptive Gauss-Legendre integration.  In essence, each of the
    functions computes the integral
    \placeformula[eq:integral]
    \startformula
      I = \int_a^b f(x) w(x) \, dx
    \stopformula
    where $w(x)$ is a weight function.  We will only be focussing on
    the case where the weight function $w(x)=1$.  Since an integral
    cannot be solved exactly by a computer the user has to provide
    error bounds to indicate convergence.

  \stopsection

  \startsection[title={Interfacing with the GSL}]

    The first thing to do when we want to interface with an external
    library is loading the \acro{FFI} modules and using it to load the
    shared library of interest into memory.
    \startbuffer[ffi-setup]
      local ffi = require("ffi")
      local gsl = ffi.load("gsl")
    \stopbuffer
    \ctxluabuffer[ffi-setup]
    \typebuffer[ffi-setup][option=lua]

    \startsubsection[title={C declarations}]

      Next we have to add all the C declarations which are important
      for us.  Let us first have a look over the code and then discuss
      why I wrote things the way they are.

      \startbuffer[cdefs]
        ffi.cdef[[

        typedef double(*gsl_cb)(double x, void *);

        typedef struct {
          gsl_cb F;
          void *params;
        } gsl_function;

        typedef void gsl_integration_workspace;

        gsl_integration_workspace *
         gsl_integration_workspace_alloc(size_t n);

        void gsl_integration_workspace_free(
          gsl_integration_workspace * w);

        int gsl_integration_qagiu(
          gsl_function *f,
          double a,
          double epsabs,
          double epsrel,
          size_t limit,
          gsl_integration_workspace *workspace,
          double *result,
          double *abserr);
        ]]
      \stopbuffer
      \ctxluabuffer[cdefs]
      \typebuffer[cdefs][option=lua]

      The first declaration introduces a new type, which I call
      \type{gsl_cb}, which stands for \acro{GSL} callback.  It is a
      pointer to a function which takes a floating point number and a
      void pointer and returns another floating point number.  In
      reality, this function pointer will point to a Lua function
      representing the integrand, i.e.\ $f(x)$ in
      \in{Eq.}[eq:integral].  We can ignore the unnamed second
      argument (\type{void *}) here because this is only relevant for
      the C interface of the \acro{GSL} but we still have to declare
      it.

      The next declaration is another type declaration, this time with
      the name \type{gsl_function}.  It is a structure containing two
      values; the first is the function pointer to the integrand
      \type{F}, the second a pointer to some memory where parameters
      could be located.  In our case we will not use the
      field \type{params} but we nevertheless have to declare it.
      What is very important is that the order of the fields in the
      structure is \emph{exactly the same} as in the C header file.
      Otherwise the memory alignment of the field will be off and a
      segementation fault will occurr.

      The last type declaration is for the identifier
      \type{gsl_integration_workspace}, which I simply make an alias
      for \type{void}.  Actually, when we look in the C header file of
      the \acro{GSL}, we find that the type
      \type{gsl_integration_workspace} is defined as a structure with
      several fields, so why do we not declare those fields?  The
      reason is simple: we don't care.  As you will see we do not
      access any fields of \type{gsl_integration_workspace} from the
      Lua level and the GSL library already knows what the fields are.
      Therefore I decided to make \type{gsl_integration_workspace}
      \emph{opaque}.

      The next three declarations are all function declarations which
      are straight copied from the header file:
      \type{gsl_integration_workspace_alloc} allocates enough memory
      to perform integration using \type{n} subintervals;
      \type{gsl_integration_workspace_free} releases that memory back
      to the system;  and the third function declaration,
      \type{gsl_integration_qagiu} is the actual integration routine.
      It computes the integral of the function \type{f} over the
      semi-infinite interval from~\type{a} to~$\infty$ with the
      desired absolute and relative error limits \type{epsabs} and
      \type{epsrel} using at most \type{limit} subintervals which
      have been previously allocated in \type{workspace}.  The final
      approximation and the corresponding absolute error are returned
      in \type{result} and \type{abserr}~\citation[qag].

    \stopsubsection

    \startsubsection[title={Lua interface}]

      Now that we've declared all of the library functions it is time
      that we integrate this with Lua.  To this end we write a
      function which nicely encapsulates all the lower level
      structure.  The function is named \type{gsl_qagiu} and it takes
      as parameters a (Lua) function~\type{f} (which takes one argument),
      the lower limit of the integral~\type{a}, and three optional
      arguments, the absolute error \type{epsabs}, the relative error
      \type{epsrel}, and the maximum number of subintervals~\type{N}.
      
      \startbuffer[gsl-interface]
        local gsl_f  = ffi.new("gsl_function")
        local result = ffi.new("double[1]")
        local abserr = ffi.new("double[1]")

        function gsl_qagiu(f,a,epsabs,epsrel,N)
          local N = N or 50
          local epsabs = epsabs or 1e-8
          local epsrel = epsrel or 1e-8

          gsl_f.F = ffi.cast("gsl_cb",f)
          gsl_f.params = nil

          local w =
             gsl.gsl_integration_workspace_alloc(N)

          gsl.gsl_integration_qagiu(gsl_f, a,
             epsabs, epsrel, N,
             w, result, abserr)

          gsl.gsl_integration_workspace_free(w)

          gsl_f.F:free()

          return result[0]
        end
      \stopbuffer
      \ctxluabuffer[gsl-interface]
      \typebuffer[gsl-interface][option=lua]

      We start by defining some local variables outside the function
      for better performance.  We instantiate a new value of type
      \type{gsl_function} and two arrays of length one using the
      \type{ffi.new} method.

      After processing the optional arguments, we set the fields
      \type{F} and \type{params}.  This is where it gets interesting.
      Recall that the type of the field~\type{F} is a pointer to a
      function which takes two arguments.  Even though the Lua
      function \type{f} only takes one argument we can use it
      directly, because of the way Lua deals with optional arguments.
      If the number of arguments is less than than the number of
      parameters passed to the function call, all the additional
      parameters are simply dropped.  The only problem that we have
      left is the fact that this is clearly a Lua function and not a C
      function.  To this end we use \type{ffi.cast} to cast the Lua
      function into a C function.  It can also be converted implicity
      by simply assigning~\type{f} but then it is less obvious to see
      what is going on.  At this point it is very important that the
      types of the arguments and the return value match, otherwise we
      will run into memory problems.  Because the field \type{params}
      is unused we simply set it to the null pointer by assigning
      \type{nil}. (We could probably leave it unset but that is bad
      practice.  Always initialize your variables!)

      The result and the absolute error of the integration are
      returned as output arguments from the GSL function, i.e.\ the
      variables have to have pointer type.  The easiest way to create
      a pointer to a value is by creating an array of length one of
      the desired type which we already did outside the function.
      Arrays can be implicitly cast into pointers but at the same time
      live on the stack, so we do not have to worry about heap
      allocation and deallocation.

      Next we use the previously declared functions to first allocate
      a workspace structure of sufficient size, then call the
      integration function with all of our arguments, releasing the
      workspace memory back to the system.  You might notice that not
      all of the variables in the call to the integration routine have
      been created using \type{ffi.new}.  This is indeed not necessary
      because the \acro{FFI} will try to convert Lua values to native
      C types for you implicitly.  Roughly speaking, you only have to
      use \type{ffi.new} for non-fundamental types or arrays.

      There is one last sublety to take care of.  The library function
      to which we passed the function pointer is allowed to store that
      pointer for later use.  Therefore this pointer will not decrease
      its reference count after exiting the function and can therefore
      never be garbage collected.  We are probably not going to call
      this function so many times that this memory leak will have a
      huge impact but it is certainly good practice to release
      resources on exit, so we indicate to the garbage collector that
      this pointer can be cleaned up by calling its \type{free()}
      method.

      Finally we return the result which is stored in the first
      element of the array.  Note that C uses zero-based indexing.
      
    \stopsubsection

    \startsubsection[title={Usage in pgfplots}]

      So far we have been only implementing some kind of abstract
      skeleton for numerical integration.  Now it is definitely time
      to actually use it.  To this end we aim to plot the following
      complete Fermi-Dirac integral.
      \placeformula
      \startformula
        F_{1/2}(t) = \int_0^\infty \frac{x^{1/2}}{e^{x-t}+1} dx .
      \stopformula
      What we will do now is call the \type{gsl_qagiu} routine with
      the integrand as the first argument and the lower limit as the
      second argument.  Because we want to obtain the result of the
      integration in \TeX\ we do not return the result of the
      integration but feed it back to \TeX\ using \type{tex.sprint}.

      \startbuffer[pgfplots-prototype]
        function F_one_half(t)
          tex.sprint(gsl_qagiu(function(x)
              return math.sqrt(x)/(math.exp(x-t)+1)
            end, 0))
        end
      \stopbuffer
      \ctxluabuffer[pgfplots-prototype]
      \typebuffer[pgfplots-prototype][option=lua]

      The last thing to do is plotting this function using
      \type{pgfplots}.  In the following I use \ConTeXt\ syntax but
      the \TeX\ and \LaTeX\ syntax is very similar.  It should be
      noted at this point, that for the \acro{FFI} to work in \LaTeX,
      the \type{--shell-escape} option has to be enabled, because
      these operations are considered unsafe.  First of all we need to
      tell Ti\italic{k}Z about the Lua function.  We do this using
      \type{declare function} and simply calling the Lua function with
      the argument (A \LaTeX\ user would use \type{\directlua} instead
      of \type{\ctxlua}).  There is still a slight problem.  The
      \type{pgfplots} package uses its own representation for floating
      point numbers, called \type{fpu} \citation[pgf-fpu], which is
      not compatible with Lua.  There are ways to work around this
      (see the Appendix), but the simplest solution for the moment is
      turning off the \type{fpu} for this plot.

      \startbuffer[pgfplots-example]
        \starttikzpicture[
          declare function={
            F_one_half(\t) = \ctxlua{F_one_half(\t)};
          }
        ]
          \startaxis[
            use fpu=false, % very important!
            width=6cm, no marks, samples=101,
            xlabel={$t$}, ylabel={$F_{1/2}(t)$},
          ]
            \addplot {F_one_half(x)};
          \stopaxis
        \stoptikzpicture
      \stopbuffer
      \typebuffer[pgfplots-example][option=tex]
      \startalignment[middle]
        \dontleavehmode\forcecolorhack % TikZ and ConTeXt are not good friends
        \getbuffer[pgfplots-example]
      \stopalignment

    \stopsubsection

  \stopsection

  \startsection[title={Conclusion}]

    The availability of \acro{FFI} in \LUATEX\ takes document
    processing to a completely new level.  The possibility to
    interface with native C libraries can be used to perform tasks
    which were previously untracktable, like the numerical integration
    in this tutorial.  This article was inspired by a question asked
    on Stack Exchange where a minimal working example of the
    techniques presented here can be
    found~\citation[FermiDiracIntegral].  Another example would be the
    conversion of an image from \acro{SVG} format to \acro{PDF}
    without the generation of intermediate files, as I demonstrated in
    \citation[Rsvg] using the Cairo and Rsvg-2 libraries, and recently
    Aditya Mahajan published an article on his \CONTEXT\ blog on how
    to interface the Julia programming language with \LUATEX\ via the
    \acro{FFI} \citation[Julia].
    
  \stopsection

  \startsection[title={Appendix},reference={sec:appendix}]

    During the preparation of this manuscript I was made aware, by
    Aditya Mahajan, that the approach of turing off the \type{fpu} is
    not always a viable workaround; it can fail, for instance when
    trying to plot in logscale.  Therefore one has to convert the
    function argument from \type{fpu} float to Lua number and the
    result from Lua number to \type{fpu} float.  Fortunately
    \acro{PGF} provides macros to facilitate this conversion.  Using
    those one can declare the function from the main text like such:
    \startbuffer[fpu-conversion-tex]
      \pgfmathdeclarefunction{F_one_half}{1}{%
        \pgfmathfloatparsenumber{%
          \ctxlua{
            F_one_half(\pgfmathfloatvalueof{#1})
          }%
        }%
      }
    \stopbuffer
    \getbuffer[fpu-conversion-tex] % just to check it's working
    \typebuffer[fpu-conversion-tex][option=tex]

    One does not necessarily have to rely on the macro level here.
    Since version 3 the \acro{PGF} package comes with a Lua backend for
    function evaluations which provides parser functions for
    \type{fpu} types.  With this, one could adapt the Lua function
    from the main text as follows:
    \startbuffer[fpu-conversion-lua]
      local plf = require"pgf.luamath.functions"

      function F_one_half(t)
        local t = plf.tonumber(t)
        local result = gsl_qagiu(function(x)
           return math.sqrt(x)/(math.exp(x-t)+1)
          end, 0))
        tex.sprint(plf.toTeXstring(result))
      end
    \stopbuffer
    \ctxluabuffer[fpu-conversion-lua] % just to check it's working
    \typebuffer[fpu-conversion-lua][option=lua]

  \stopsection

  \startsubject[title={References}]
    \placelistofpublications
  \stopsubject
  
\StopArticle
\stoptext


%https://tug.org/pipermail/luatex/2017-February/006345.html
